\documentclass[biblist]{article}
\usepackage{amsfonts, amsmath, amssymb, amsthm}
\usepackage{hyperref, graphicx, verbatim, listings, multirow}
\usepackage{algorithm, algorithmic}
% \usepackage{fancyhdr}

% \ifdefined\makeindex
%   \message{\string\makeindex\space is defined}\usepackage{natbib}\setcitestyle{super}%
% \else
%   \message{no command \string\makeindex}%
% \fi

% To remove excessive vertical spacing while using itemize etc.
\usepackage{enumitem}
\setlist{nolistsep}

% Include and configure tikz
\usepackage{tikz}
\usetikzlibrary{arrows,decorations,fit,backgrounds}

\tikzstyle{gm-var-constant}=[rectangle, draw, fill=black!30, thick, minimum size=1mm]
\tikzstyle{gm-var-hidden}=[circle, draw, thick, minimum size=1mm]
\tikzstyle{gm-var-seen}=[gm-var-hidden, fill=black!30, minimum size=1mm]
\tikzstyle{gm-plate} = [thick,draw=black,rounded corners=2mm]

% For clique trees.
\tikzstyle{ct-shared-nodes}=[rectangle, draw, fill=black!30, thick, minimum size=1mm]
\tikzstyle{ct-clique}=[circle, draw, thick, minimum size=1mm]


% \usepackage[bottom]{footmisc}

% COnfigure the listings package: break lines.
\lstset{breaklines=true}

\setcounter{tocdepth}{3}

% Lets verbatim and verb environments automatically break lines.
\makeatletter
\def\@xobeysp{ }
\makeatother
% \lstset{breaklines=true,basicstyle=\ttfamily}

% Packages not included:
% For multiline comments, use caption package. But this conflicts with hyperref while making html files.
% subfigure conflicts with use with memoir style-sheet.

\usepackage[empty]{fullpage}
\usepackage{microtype} % Full justification
\usepackage{natbib}
\usepackage{enumitem} % Compact lists
\setlist{noitemsep,topsep=0pt,parsep=0pt,partopsep=0pt}% parsep affects inter-item spacing, topsep affects space after the list.

\definecolor{mygrey}{gray}{.95}

% Adjust margins
\addtolength{\oddsidemargin}{-0.375in}
\addtolength{\evensidemargin}{0.375in}
\addtolength{\textwidth}{0.5in}
\addtolength{\topmargin}{-.375in}
\addtolength{\textheight}{0.75in}
\parindent 0cm
\parskip 0.25\baselineskip

\renewcommand{\section}[1]{
% The following two lines should appear in sequence.
\phantomsection%
\addcontentsline{toc}{chapter}{#1}%
\pagebreak[3]%
\hyphenpenalty=10000%
\vspace{0.75\baselineskip}%
\raggedright{\colorbox{mygrey}{\textbf{#1}}}%
\\\vspace{-0.7\baselineskip}\hrulefill%
}



\newcommand{\resheading}[4]{
\begin{tabular*}{\columnwidth}{@{}l@{\extracolsep{\fill}}r@{}}
\textbf{#1} & #2 \\
#3 & #4 \\
\end{tabular*}}

% \newenvironment{itemize*}%
%   {\begin{itemize}%
%     \setlength{\itemsep}{0pt}%
%     \setlength{\parskip}{0pt}}%
%   {\end{itemize}}



\newcommand{\headeditem}[1]{
    \item\textbf{#1}  }
\newcommand{\headedsubitem}[1]{
    \subitem\textbf{#1}  }

\newcommand{\nameDisplay}[1]{
\center{\textbf{\large{#1}}}\\\hrulefill}

\begin{document}
\nameDisplay{Vishvas Vasuki}

\begin{tabular*}{\columnwidth}{@{}l@{\extracolsep{\fill}}r@{}}
\parbox{2in}{\footnotesize \textbf{Address} Mountain View, CA, USA\\
\small \url{https://vvasuki.github.io/}\\} &
\parbox{3in}{\footnotesize Email: vishvas.vasuki [\ae{}] gmail.com\\
%Phone: +1 $2^9$ 524 6901\\
}
\end{tabular*}

% \begin{resume}

\begin{comment}
\section{Summary}
\begin{itemize}
 \item \textbf{As a researcher, tool-builder, publisher}: Interest and past work in Sanskrit natural language processing.
 \item \textbf{As software engineer}: Rich experience as a software engineer at top companies.
\end{itemize}
\end{comment}



\vspace{-16pt}
\section{Formal Education}
\resheading{\href{http://www.cs.utexas.edu}{University of Texas Austin}}{Computer Science}{MS with GPA 3.7.}{Dec, 2010}
\begin{comment}
\begin{itemize}
 \item \textit{Relevant coursework:} \href{http://www.cs.utexas.edu/~klivans/colt08.html}{Computational Learning Theory}, \href{http://www.cs.utexas.edu/users/inderjit/courses/cs383c_2008/}{Numerical Analysis: Linear Algebra},  \href{http://users.ece.utexas.edu/~sanghavi/spring2010_course.pdf}{Sparsity Structure and Algorithms}, \href{http://www.cs.utexas.edu/~plaxton/c/388r/index.html}{Randomized Algorithms}, Nonlinear Programming, \href{http://www.cs.utexas.edu/users/inderjit/courses/dm2009/}{Data Mining}.
\end{itemize}
\end{comment}

\begin{comment}
% \href{http://vishvas-vasuki.appspot.com/images/documents/utexasDegreeChkPhD.pdf}{\textit{Courses taken for credit\\}}
% \href{http://www.cs.utexas.edu/users/panni/cs388T.html}{Theory of Computation},
%  \href{http://www.cs.utexas.edu/~bwaters/classes/cs395t-spring09/}{Advanced Cryptography}, \href{http://www.cs.utexas.edu/~acharya/}{Distributed Computing}.
%
% \textit{Courses audited\\}
% \href{http://math.utexas.edu/users/henrys/m365c.html} {Real Analysis 1}, \href{http://ssc.utexas.edu/worddocs/Duthie_M378K_Syllabus.pdf} {Introduction to Mathematical Statistics},
% \href{http://www.cs.utexas.edu/~pradeepr/courses/395T-LT/}{Learning Theory}.

\end{comment}

\resheading{\href{http://www.vtu.ac.in}{Vishveshvariah Technological University}, India}{Computer Science and Engineering}{BE. \textit{First class with distinction.}}{Jul, 2003}

\begin{comment}

\subsubsection{Non-CS Courses}
Engineering mathematics, Elements of Civil Engineering, Strength Of Materials, Engineering Chemistry, Engineering Physics, Electrical Science, Mechanical Engineering, Engineering Graphics, Workshop Practice.

\subsubsection{Non-technical Courses}
Computing Profession And Society, Management in Engineering.
\end{comment}

\section{Publications}
\vspace{-8pt}
\renewcommand{\refname}{}
\renewcommand{\bibsection}{}
\setlength{\bibsep}{0pt}
\verb' '\nocite{*}
\bibliographystyle{unsrtnat}
\bibliography{proceedings}

\section{Research}
\resheading{\href{}{Sanskrit and Hindu community}}
{}
{Independent Researcher}
{2011 to present}
Sanskrit language processing.
\textit{Highlights (Core):} Published several indic language stardict dictionaries, a mobile app to conveniently install them. An easy to use audio podcasting service. Libraries for audio and video curation. Indic transliteration modules (scala, python). The Jekyll sanskrit-documentation-theme.
\textit{Highlights (Secondary):} Open source Jyotisha library, sanskrit\_parser (scraped data), m17n transliteration library.
Consolidated the \textit{open source \href{https://sanskrit-coders.github.io/site/}{Sanskrit programming community}} with an active mailing list.

\resheading{\href{}{MIT-SVS}}
{}
{Researcher}
{Apr 2017 to Mar 2018}
Sai's "Vedavaapi" initiative: "ullekhanam" (text and image annotation), "vyakarana" (sanskrit grammar) projects.

\resheading{\href{http://www.honda-ri.com/HRI_Us/}{Honda Research Institute USA} via TriCom Quest}
{Mountain View, CA}
{Research Intern}
{Jan 2012 to Jun 2012}
% {16 Jan 2012 to 1 Jun 2012}
\textit{Product prototyped:} Spoken Dialog System for interactive navigation.
\textit{Activities:} Scaling-up the parser to work with large database of entities (cut running time from O(n) to O(1)).
% A monolithic FST with O(n) nodes was replaced by an FST with constant number of nodes transducing the output of a trie-based entity tagger.
Tuning belief tracking system. Assisted with demostrating this complex system on an Android device. Probabilistic graphical models and neural networks were used. Involved working with Java, Scala, C and Python code.

\resheading{\href{http://www.cs.utexas.edu}{University of Texas Austin}}
{Department of Linguistics}
{Volunteer Research Assistant}
{Jun 2011 to Jan 2012}
% {20 Jun 2011 to 15 Jan 2012}
\textit{Topic:} Bootstrapping a Part of speech tagging using tag dictionaries.
\textit{Techniques:} Modeled context using Directed graphs and Hidden Markov Models. Used EM with Label Propagation to exploit untagged data.

\resheading{\href{http://www.cs.utexas.edu}{University of Texas Austin}}
{Department of Computer Sciences}
{Research Assistant}
{Jul 2009 to Dec 2010}
\textit{Problems considered:} Learning discrete graphical models, devising realistic network generation models, building recommender systems which use information from auxiliary social networks, gene-disease link discovery. \\
\textit{Activities included:} Literature survey, identifying problems of interest, modeling and some theoretical analysis, designing experiments, tinkering and finding solutions, prototyping algorithms, presenting ideas and results, assisting in grant applications. The research involved collaboration and some mentoring.



\resheading{\href{http://cs.umass.edu}{University of Massachusetts Amherst}}
{Computer Science Department}
{Research Assistant}
{Sep 2007 to Dec 2007}
\textit{Topic:} Use of homology information in improving ab-initio Protein Structure Prediction. \\
\textit{My role involved:} Literature survey, developing a research plan and contributing to the prediction software.


\resheading{\href{http://ece.iisc.ernet.in}{Indian Institute of Science}}
{Electrical Communications Engineering}
{Project Assistant}
{Nov 2006 to Mar 2007}
% 6/11/2006	to 3/3/2007
\textit{Topic:} The use of wireless sensor networks in detecting, classifying and tracking various objects. \\
\textit{My role:} To evaluate sensing, routing and scheduling algorithms by simulation, and by actual deployment.


\section{Software engineering}
\resheading{\href{http://www.youtube.com/}{Google Inc.}}
{YouTube}
{Software Engineer}
{Jun 2012 to Apr 2017}
% {04 Jun 2012 to 3 Apr 2017}
\textit{Product:} YouTube Spam and Abuse classifiers, Ad inventory classifiers. \\
\textit{Role:} I built, evaluated and deployed several machine learning models. Mostly text features, linear models and neural networks (Google Brain). \textit{Auxiliary initiatives:} Pointed out and fixed problems in Indic language products (Eg. OCR, Google indic IME-s).\\

\resheading{\href{http://www.microsoft.com/india}{Microsoft}}
{Windows Group}
{Software Development Engineer}
{Feb 2006 to Aug 2006}
% {13 Feb 2006 to 31 Aug 2006}
\textit{Product:} Java terminal services client deployed on \textbf{Microsoft Windows Storage Server 2003}. Starting from scratch, we delivered an early version of the product in six months. \\
\textit{Role:} I developed and owned the input-handling and packet-framing modules.


\resheading{\href{http://www.infosys.com}{Infosys}}
{PLM Group}
{Software Engineer}
{Jul 2003 to Dec 2004}
% {14 Jul 2003 to 15 Dec 2004}
\textit{Products:} Ematrix's Product Central, a machine-part management system for Porsche.\\
\textit{Role:} I implemented fresh features, fixed bugs, did some testing, produced technical documentation, maintained the bug-tracking (bugzilla) portal, made UI mock-ups, participated in design, and reviewed code. \textit{Self-driven initiatives:} Writing software for bug-notification, synchronizing bug databases and ranking employees based on minimal input.


\begin{comment}
\resheading{\href{http://www.satyam.com}{\bf Satyam Computer Services Ltd.}}
{Project Trainee}
{Bengaluru, India}
{Feb 2003 to Jul 2003}

We wrote a fast spreadsheet program using VB which worked with both XML and Excel formats. This software also prevented cyclical reference of cells in arithmetic formulae.

\end{comment}

\section{Technical Skills}
\begin{itemize}
\headeditem{Recent Experience:}
 Python, C++, Scala, Java.
\headeditem{Past Experience:}
\small{
Matlab, R, \LaTeX, C, C\#, VB, C++, 8086 assembly language, NesC/ TinyOS, JSP, Rational Rose, SQL, E-Matrix's PLM suite, JavaScript, PERL. \textbf{Web-server environments:}  Apache, Weblogic, Websphere, Tomcat. \textbf{Frameworks and standards:}  log4j, XML, CSS, JUNIT, ANT.}
\headeditem{Code samples} See github repos under \href{https://github.com/vvasuki}{vvasuki}, \href{https://github.com/sanskrit-coders}{sanskrit-coders}.
\end{itemize}

\section{Honors and Scholastic achievements}
\begin{itemize}
\headeditem{Merit scholarship} \small{from the State Bank of Mysore. \emph{Microelectronics and Computer Development Fellowship} UT 2007 (Declined). \emph{Deans Excellence Award} UT 2007 (Declined).}

\headeditem{Standardised tests:}
\headedsubitem{GRE 09/2011} \small{Verbal: \textbf{170/170}, Quant: \textbf{166}. Analytical Writing: \textbf{4.5/6}\tiny{ - 99, 94, 72-th \%iles.}}
\headedsubitem{GRE 07/2006} \small{Verbal, Quant: \textbf{800/800}. Analytical Writing: \textbf{5.5/6}. \tiny{ -99, 94, 96th \%iles.}}
% \headedsubitem{TOEFL CBT test - 2006: 300/300}
\headedsubitem{Computer Science GATE 2007: 99th percentile [Rank 181]} \footnote{This annual national level exam tests computer science knowledge in over 17,000 (in 2007) candidates to help postgraduate admissions. }
\item \emph{madhyamA} in Sanskrit and Hindi.
\end{itemize}

\section{Teaching}
\resheading{\href{http://www.cs.utexas.edu}{University of Texas Austin}}
{Department of Computer Sciences}
{Teaching Assistant}
{Jan 2008 to Dec 2010}
My duties have included grading, making grading rubrics, proctoring exams, holding office hours and leading discussion sections. I was TA for the following courses:
\textit{Learning Theory}(\href{http://www.cs.utexas.edu/~pradeepr/courses/395T-LT/}{CS395T}), Fall '10. \textit{Numerical Analysis: Linear Algebra}(\href{http://www.cs.utexas.edu/\%7Ecline/CS383C/}{CS383C}), Fall '09. \textit{Contemporary issues in Computer Science} (\href{http://www.cs.utexas.edu/~ear/cs349/index.html}{CS349}), Spr '09. \textit{Analysis of Programs} (\href{http://www.cs.utexas.edu/\%7Ecline/CS336/}{CS336}), Fall '08. \textit{Foundations of Computer Science} (\href{http://www.cs.utexas.edu/~scottm/cs307/}{CS307}), Spr '08.


\begin{comment}
\section{Talks}
\begin{itemize}
\item \href{https://raw.githubusercontent.com/vvasuki/notes/master/graphTheory/communityRecommendation/talkNaga/affiliationRecommendation.pdf}{Affiliation recommendation using auxiliary social networks}, RPE, Fall 2010, UT Austin.
\item \href{https://raw.githubusercontent.com/vvasuki/notes/master/optimization/leastAnglesRegression/leastAnglesRegression.pdf}{Least Angle Regression algorithm}, Data Mining course, Fall 2009, UT Austin.
\item \href{https://raw.githubusercontent.com/vvasuki/notes/master/linAlg/exposition/kyFanNorms/kyFanNorms.pdf}{Ky Fan (p, k) norms}, April 2009, Data mining lab, UT Austin.
\end{itemize}

\section{Service and Membership in Professional Societies}
\begin{itemize}
\headeditem{SIAM: Society for Industrial and Applied Mathematics} Fall 2008 to Fall 2010.
\headedsubitem{SIAM Activity Group on Discrete Mathematics}
\headedsubitem{SIAM Activity Group on Computer Science and Engineering}

\headeditem{University of Texas Libraries Committee}: Alternate representative of the Graduate Students Assembly. Fall 2008 to 2009.

\headeditem{University of Texas Graduate Student Assembly}: Administrative officer. Summer 2009 to 2010.

\end{itemize}
\end{comment}


\section{Personal Details}
\begin{itemize}
\headeditem{Avocation:} Sanskrit, Poetry, shAstra, Nature, Hiking, Vedic conservativism.
\headeditem{Human languages:} English, Spoken samsk.rta, kannaDa, hebbAr tamiL, hindI.
\headeditem{Citizenship:} India.
\end{itemize}


\begin{comment}
\section{References}
\begin{itemize}
\item Nobuyuki Morioka, Charlie Curry Google.
\item \href{http://www.linkedin.com/in/venumoram}{Venugopala Rao Moram}, Senior SDE, Microsoft, WA. Email: venugopala.moram [\ae{}] microsoft.com
\item \href{http://www.linkedin.com/pub/manoj-thomas/7/29/81a}{Manoj Thomas}, Principle member of Technical Staff, Oracle, CA. Email: manojpt [\ae{}] gmail.com
\item \href{http://www.linkedin.com/pub/zhengdong-lu/0/167/ab4}{Zhengdong Lu}, Researcher, Microsoft Research Asia, Beijing. Email: lzhengdong [\ae{}] gmail.com, zhengdol [\ae{}] microsoft.com
\end{itemize}
\end{comment}

% \end{resume}

\end{document}
